\section{Evaluating GAN performance}
Evaluating performance of a discriminative model is a rather straightforward issue. One would reserve a portion of data available for a test data set and after training run the model on this data set and count the number of samples correctly classified. With generative models things get more complicated. Judging the quality of generated samples, whether these are images, text or something else, can be challenging and often subjective. However, this has to be done in order to compare different GANs regular;y proposed by researches and therefore some creative ideas were proposed in this field. Some of the method can be applied only in a particular domain, like generating images with multiple objects on them, while others can be applied to an arbitrary GAN. 
\subsection{Manual evaluation}
The best way to judge the quality of generated contend is still by using real people. 
\subsection{Inception score}

\subsection{Generative adversarial metric(GAM)}
This method provides relative score when comparing several generative adversarial networks.
