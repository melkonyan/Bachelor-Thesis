\section{Introcuction}
After their introduction in 2014 by Ian Goodfellow generative adversarial networks(GANs) have become a subject of an intensive research~\citep{gan}. Generative adversarial networks belong, as the name suggests, to the class of generative models. Such models try to learn the underlying probability distribution of an observed data. Once learned, this distribution can be used to generate new, realistic samples of data, which, depending on dataset, could be an image with a human face on it, a piano melody or a piece of poetry. Generative adversarial networks also introduced us to new problems, not observed by discriminative models. The first one of these problems is unstable training. GANs seem to be very sensitive to hyper-parameters and a network architecture. The newly proposed WassersetinGAN(WGAN) aims to relax this sensitivity by providing a new type of loss function, which allows to train a wider variety of networks and spend less time finding a perfect set of hyper-parameters. Another problem inherent to GANs is that the quality of the data produced by a network is hard to evaluate automatically. This together with the fact that new GAN architectures emerge weekly makes it hard to compare all of them to choose the state-of-the-art one. In the case of discriminative models with labeled data, like image classification, evaluation of network performance is a straightforward task. One could just run a network on a test dataset and count the number of correctly classified images. This is not possible with generative models, that produce some new, unseen data. Judging the quality of this data is hard to automate and sometimes it is easier to do it manually. Of coarse, the problem with manual approach is that it will is not fully objective and does not scale.\\ 
\indent In this thesis I have implemented two Deep convolutional generative adversarial networks(DCGANs) with both cross-entropy and Wasserstein loss functions respectively. Then I tried to compare their performance by variation of a Generative adversarial metric(GAM). 